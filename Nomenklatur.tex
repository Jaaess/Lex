%\lohead[]{llll}
\chapter{Nomenklatur}
    \section{Formelzeichen}
    \begin{longtable}[l]{p{3cm}|p{12cm}}
        Formelzeichen & Beschreibung\\
        \hline
        $A$ &Ausgabeschicht/Ruhelage\\
        $c$ & Koeffizienten im Polynom/Koeffizienten in Teststrecke \\
        $\vec{c}$ &Center-Vektor\\
        $\vec{d}$ & Abstand zwischen zwei Vektoren\\
        $E$ & Eingabeschicht/Fehler\\
        $e$ & Quadratischer Fehler\\
        $f$ & Funktion/Frequenz\\ 
        $flops$ & Fließkommaoperationen \\
        $g$ & Funktion/Parameter\\
        $H$ & Verborgene Schicht bzw. Layer\\
        $h$ & Funktion\\
        $I$ & Netzeingabe\\
        $|I|$ & Anzahl Beispiele\\
        $\vec{i}$ & Input eines einzelnen Neurons\\
        $J$ & Jacobimatrix\\
        $k$ & Anzahl \\
        $m$ & Anzahl der Ausgänge/ganzzahliges Vielfaches\\
        $N$ & Menge aller Neuronen\\
        $n$ & Anzahl Neuronen/Anzahl Trainingsbeispiele\\
        $O$ & Netzausgabe\\
        $o$ & Output eines einzelnen Neurons\\
        $P$ & Performance\\
        $p$ & Anzahl der Beispiele die erneuert werden\\
        $q$ & Grad des Polynoms/Anzahl der Layer\\
        $r$ & Zeitauflösung\\
        $s$ & Grenze bzw. Schranke\\
        $T$ & Ausgabemuster/Teachinginput\\
        $t$ & Zeit bzw. Zeitpunkt\\
        $u$ & Stellgröße\\
        $V$ & Menge aller Verbindungen\\
        $v$ & Gewicht/Polynom\\
        $W$ & Matrix aller Gewichte\\
        $\vec{w}$ & Eingangsgewichte eines einzelnen Neurons\\
        $w$ & Gewicht\\
        $Y$ & Messwert\\
        $y$ & Regelgröße\\
        $\vec{Z}$ & Innere Zustände eines Neurons\\
        &\\
        $\alpha$ & Parameter \\
        $\beta$ & Parameter \\
        $\Delta$ & Änderung \\
        $\delta$ & Gewichtsänderung\\
        $\eta$ & Lernrate\\
        $\Theta$ & Parameter\\
        $\mu$ & Regularisierungsfaktor\\
        $\rho$ & Korrelationswert\\ 
        $\tau$ & Integrationsvariable\\
        $\nabla$ & Nablaoperator \\
       
    \end{longtable}

    \section{Indizes}
    
    \begin{longtable}[l]{p{3cm}|p{12cm}}
        Index & Beschreibung\\
        \hline
        $A$ & Aktivierung\\
        $alt$ & alt/aus vergangenem Zeitschritt\\
        $i$ & Zählindex\\
        $j$ & Zählindex\\
        $k$ & Zählindex\\
        $last$ & Vergangene Werte\\
        $Modell$/$M$ & Größen die aus dem Modell berechnet wurden\\
        $max$ & Maximal\\
        $o$ & Ausgabe\\
        $P$ & Propagierung\\
        $p$ & Prediction\\
        $^p$ & Ein bestimmtes Trainingsmuster\\
        $q$ & Zählindex\\
        $rand$ & Zufällig\\  
        $s$ & Statisch\\
        $T$ & Teachinginput \\
        $Train$ & Training\\
        $Val$ & Validation\\            
        
        
    \end{longtable}
