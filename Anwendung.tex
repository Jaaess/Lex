\chapter{Anwendung}
    Hier ein neues Kapitel
    Viele Zitate: \cite{patterson} \cite{krizhevsky} \cite{matlab} \cite{pitts} \cite{lawrence} \cite{miesbach}
    \section{System Bedienung}
        Eine Section
        \subsection{Subsection}
        
        
Hier ist ein Bild:        
            \begin{figure}[h]
                \includegraphics[scale=0.2]{Abbildungen/Kapitel4/Big-architecture.png}
                \centering
                \caption{Affe Bild}
                \label{Abb:Affe}   
            \end{figure}  
            
Hier ist eine Tabelle: text text text text text text text text text text text text text text text text text text text text text text text text text text text text text text text text text text text text text text text text text text text text text text text text text text text text text text text text text text text text text text text text text text text text text text text text text text text text text text text text text text text text text text text text text text text text text text text text text text text text text text text text text text text text text text text text text text text text text text text text text.           
            
             \begin{table}[h]
                \begin{tabular}{ccc}
                      \hline
                      Spalte1 & Spalte2 & Spalte3\\                      
                      \hline
                      1 & 2 & 3\\
                      \hline
                \end{tabular}
                \centering
                \caption{Quadratewurzel Skalierung}
                \label{Tab:Quadratewurzel Skalierung}
            \end{table}
            
Hier eine neue Tabelle: text text text text text text text text text text text text text text text text text text text text text text text text text text text text text text text text text text text text text text text text text text text text text text text text text text text text text text text text text text text text text text text text text text text text text text text text text text text text text text text text text text text text text text text text text text text text text text text text text text text text text text text text text text text text text text text text text text text text text text text text text.   
            
            \begin{table}[h]
                \begin{tabular}{cccc}
                      \hline
                      Spalte1 & Spalte2 & Spalte3 & jbjvh\\                      
                      \hline
                      1 & 2 & 3 & 4\\
                      \hline
                \end{tabular}
                \centering
                \caption{Logaritmische Skalierung}
                \label{Tab:Logaritmische Skalierung}
            \end{table}
 
        
        
    \section{gebliebene Fehlermeldungen und Ursachen}
        Eine Subsection
	\section{Nutzer Reaktion}
        Eine Subsection
  