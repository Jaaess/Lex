\chapter{Ergebnisse}
    \label{Ergebnisse}
    Im folgenden Kapitel wird das entwickelte System zur Modellidentifikation angewendet. Dafür werden wieder künstliche Messdaten verwendet. Die Ergebnisse beschränken sich 
    auf Strecken mit einem realen physikalischen Bezug. Daher wird von den Systemen aus Kapitel \ref{Analyse} nur die Strecke Nr.1 betrachtet. Zusätzlich wird der sogenannte 
    Lorenz-Attraktor als Strecke verwendet.
    \section{Künstliche Messreihen}
        Ein Test mit künstlichen Messdaten ist nur beschränkt aussagekräftig, da keine (bzw. nur künstliche) Störungen auftreten. Der Vorteil ist, dass bei einem 
        bekannten System die Ergebnisse leicht bewertet werden können. Die Ergebnisse in diesem Kapitel sollen dazu dienen, die im vorangegangenen Kapitel \ref{Aufbau} gemachten Annahmen 
        zu bestätigen bzw. zu widerlegen. Mit der ersten Strecke wird zunächst eine passende Kombination der Parameter bestimmt, bevor im nächsten Schritt mit einer zweiten Strecke
        überprüft werden soll, wie gut das eingestellte System bei einer unbekannten Strecke arbeitet.
        \subsection{Strecke Nr.1}
            Die Strecke Nr.1 wurde in Kapitel \ref{Analyse} bereits in einer leicht abgewandelten Form für die Analyse der Netzstrukturen verwendet. 
            Da in der allgemeinen Analyse nicht auf die physikalische 
            Bedeutung der Strecke eingegangen wurde, soll die Erläuterung an dieser Stelle kurz nachgeholt werden.
            \subsubsection{Streckenbeschreibung}
                Das System der Differentialgleichungen beschreibt die Strömung um einen zweidimensionalen Körper. Die Gleichungen und Überlegungen wurden aus dem 
                Vorlesungsskript \cite{kingNL} übernommen. Der Körper ist in Abbildung \ref{Abb:Koerper} dargestellt.
                \begin{figure}[h]
                    \includegraphics[scale=1.0]{Abbildungen/Ergebnisse/Koerper.png}
                    \centering
                    \caption{Profil im Windkanal (aus \cite{kingNL})}
                    \label{Abb:Koerper}    
                \end{figure}
                Am Ende des Profils kann die Strömung durch periodisches Einblasen oder Absaugen im Nachlauf beeinflusst werden. Das System wird durch die folgenden 
                Gleichungen beschrieben:
                \begin{align}
                    \begin{split}
                    &\dot{A}_1 = c_1(1-c_2\,a_3 - c_3\,a_4)A_1\\
                    &\dot{A}_2 = c_4\,A_2+g\,\tilde{u} \\
                    &\dot{a}_3 = c_5(a_3 - A_1^2) \\
                    &\dot{a}_4 = c_6(a_4 - A_2^2)
                    \end{split}
                \end{align}
                Dabei beschreiben $A_1$ und $A_2$ die Amplituden im nicht angeregten und im angeregten Zustand. Die Zustandsgrößen $a_3$ und $a_4$ entsprechen den Modenamplituden zweier
                Shiftmoden. Mit einer konstanten stationären Stellgröße $\tilde{u}_s \neq 0$ ergeben sich mehrere Ruhelagen. Eine davon ist 
                \begin{align}
                    &A_{1s} = \sqrt{\frac{1}{c_2}-\frac{c_3\,g^2}{c_2\,c_4^2}\tilde{u}_s^2}\\
                    &A_{2s} = - \frac{g}{c_4}\tilde{u}_s\\
                    &a_{3s} = \frac{1}{c_2}-\frac{c_3\,g^2}{c_2\,c_4^2}\tilde{u}_s^2\\
                    &a_{4s} = \frac{g^2}{c_4^2}\tilde{u}^2_s.
                \end{align}
                Werden nun neue Koordinaten $\vec{x} = \vec{\tilde{x}}-\vec{x_s} = (A_1,A_2,a_3,a_4)^T - (A_{1s},A_{2s},a_{3s},a_{4s})^T$ und $u = \tilde{u} - u_s$ eingeführt, welche 
                die Auslenkung von diesem stationären Arbeitspunkt beschreiben, kann das System alternativ beschrieben werden:
                \begin{align}
                \begin{split}
                    &\dot{x}_1 = c_1(c_2\,x_3 + c_3\,x_4)(A_{1s}+x_1)\\
                    &\dot{x}_2 = c_4\,x_2+g\,u \\
                    &\dot{x}_3 = c_5(2\,A_{1s}\,x_1 + x_1^2 - x_3) \\
                    &\dot{x}_4 = c_6(2\,A_{2s}\,x_2 + x_2^2 - x_4)
                    \end{split} 
                \end{align}
                Der Systemausgang ist wie bisher die Summe aus den beiden Zustandvariablen $x_1$ und $x_2$. 
                Als Parameter wurden die Werte in Tabelle \ref{Tab:Parameter1} gewählt. Als stationärer Arbeitspunkt wurde $u_s = 0{,}5$ eingestellt.
                \begin{table}[h]
                    \begin{tabular}{|c||c|c|c|c|c|c|c|}
                        \hline
                        Parameter &$c_1$&$c_2$&$c_3$&$c_4$&$c_5$&$c_6$&$g$\\
                        \hline
                        Wert &$10$&$31$&$25$&$-10$&$20$&$20$&$2$\\
                        \hline
                    \end{tabular}
                    \centering
                    \caption{Parameter der Strecke Nr.1}
                    \label{Tab:Parameter1}
                \end{table}
         
            \subsubsection{Parameteranpassung}
                Um für die Parametereinstellung einen Ausgangspunkt festzulegen, wird zunächst eine Berechnung mit einer willkürlichen Kombination der Parameter durchgeführt. Mit diesen 
                Ergebnissen als Referenz werden dann die Parameter einzeln betrachtet und angepasst. Eine Tabelle mit allen Testreihen ist in Anhang \ref{Anhang:Online_Tabelle} zu finden. Die Parameter 
                der Ausgangskonfiguration sind in Tabelle \ref{Tab:ParameterRef} aufgelistet.
                \begin{table}[h]
                    \begin{tabular}{c||ccccccc}
                        %\hline
                        Parameter &Signal&Eingänge $u(t)$&Eingänge $y(t)$&$|I|_{last}$&$|I|_{rand}$&$\Delta|I|_{rand}$&$n_{max}$\\
                        %\hline
                        Wert &Step&$3$&$3$&$10$&$100$&$1$&$5$\\
                        \hline
                        \hline
                        Parameter &$k_{p}$&$s0_{E}$&$\Delta s_{E}$&$k_{max}$&&&\\
                        %\hline
                        Wert &$5$&$10^{-3}$&$0{,}8$&$2000$&&&\\
                        %\hline
                    \end{tabular}
                    \centering
                    \caption{Parameter für die Referenzsimulation (Strecke Nr.1)}
                    \label{Tab:ParameterRef}
                \end{table}
                Um den Zeitaufwand einer Simulation zu beschränken, wurde jeweils ein Zeitraum von etwa 50 Minuten betrachtet. Der zulässige Fehler $E_{max}$ wurde mit $10^{-7}$ sehr klein gewählt,
                um sicherzustellen, dass das Training über die gesamte Simulationszeit fortgesetzt wird. Um Rechenzeit zu sparen, wird die Prediction immer nur alle 100 Zeitschritte berechnet.\\
                Zur Analyse wird der Verlauf des online berechneten Fehlermaßes über die Zeit betrachtet. Abbildung \ref{Abb:RealTime_Testreihe1} 
                zeigt den Verlauf zusammen mit der Strukturentwicklung des Netzes. 
                \begin{figure}[h]
                    \begin{tikzpicture}
                        \begin{semilogyaxis}[xmin = 0, xmax = 30, width=16cm, height=6cm, grid=major, grid style={dashed,gray!30}, xlabel=Zeit, 
                            ylabel=Fehler, x unit= Min , y unit=] 
                            \addplot[mark=none,blue] table[x index = {0}, y index = {1}]{Daten/Auswertung_Echtzeit_Setup_1/pred_error.txt};
                        \end{semilogyaxis}
                    \end{tikzpicture}
                    \begin{tikzpicture}
                        \begin{axis}[xmin = 0, xmax = 30, width=16cm, height=4cm, grid=major, grid style={dashed,gray!30}, xlabel=Zeit, 
                            ylabel=Neuronen, x unit= Min, y unit= ,]  
                            \addplot[mark=none,red] table[x index = {0}, y index = {1}]{Daten/Auswertung_Echtzeit_Setup_1/struktur.txt};
                            %\draw (axis cs:1000,2) circle[radius=2];
                        \end{axis}
                    \end{tikzpicture}
                    \caption{Fehler und Strukturverlauf (Anzahl der Neuronen auf dem letzten Layer) für Referenzsimulation (Strecke Nr.1)}
                    \label{Abb:RealTime_Testreihe1}
                \end{figure}
                Dabei sind allerdings nur die ersten 30 Minuten dargestellt, da danach keine Strukturanpassungen mehr auftreten. Dies bedeutet auch, dass sich der Fehler in 
                diesem Bereich nicht mehr verkleinert.\\
                Es lässt sich erkennen, dass sich der Fehler über die Zeit reduziert. Zwar ist dieser Vorgang nicht monoton, allerdings lassen sich deutlich die zunehmend kleineren
                Spitzen erkennen. Vor allem in den ersten vier Minuten wird die Struktur häufig angepasst, was auf eine schnelle Verbesserung der Performance schließen lässt. 
                Danach wird über mehr als 20 Minuten mit dieser Struktur trainiert, bis wieder die Fehlergrenze unterschritten wird und die Struktur erneut angepasst wird. Mit 
                dieser Struktur wird dann ein Ergebnis von etwa $5 \cdot 10^{-6}$ erzielt und es wird ein neuer Layer hinzugefügt (siehe Abbildung \ref{Abb:RealTime_Testreihe1}). Der neue Layer entsteht an 
                der Stelle wo der Wert von fünf auf eins wechselt, da nur die Anzahl der Neuronen auf dem letzten Layer aufgetragen ist. Danach kann 
                die Fehlergrenze, die nun 20\,\% unterhalb von etwa $5 \cdot 10^{-6}$ liegt, in der simulierten Zeit nicht mehr erreicht werden.\\
                Um Veränderungen mit den einzelnen Parametern zu identifizieren, wird im Folgenden jeweils der entsprechende Parameter verändert. Dabei war es durch den zeitlichen 
                Aufwand im Rahmen dieser Arbeit nicht möglich, die Parameter so ausführlich zu variieren wie in der Offline-Betrachtung. Alle Testreihen für den Online-Betrieb sind 
                in Anhang \ref{Anhang:Online_Tabelle} aufgelistet. Die jeweiligen Verläufe des Fehlers und 
                der Strukturanpassungen sind ebenfalls in Anhang \ref{Anhang:Testreihen_online} zu finden. An dieser Stelle werden die Ergebnisse kurz zusammengefasst:
                \begin{itemize}
                    \item \textbf{Alternatives Testsignal:} Für das Training z.B. mit einem Chirp-Signal ergeben sich ähnliche Verläufe des Fehlers und
                    der Strukturentwicklung. Es ist davon auszugehen, dass im Trainingsbetrieb das Testsignal kaum einen Einfluss auf die Identifikation hat. Ob 
                    das Modell nach Abschluss des Trainings mit allen Eingangssignalen gleich gut umgehen kann, ist getrennt zu bewerten.
                    \item \textbf{Anzahl vergangener Beispiele $|I|_{last}$:} Eine höhere Anzahl an Beispielen aus direkt voran gegangenen Zeitschritten scheint in dem durchgeführten 
                    Test zu einer schnelleren Strukturanpassung zu führen. Ein gutes Ergebnis in Bezug auf den Fehler wird bereits bei etwa sieben Minuten erreicht. Danach ergibt sich auch 
                    hier keine Verbesserung mehr. Generell ist davon auszugehen, dass eine höhere Anzahl an Beispielen aus direkt vorangegangenen Zeitschritten zu einer besseren 
                    Anpassung auf den aktuellen Zustand führt. Damit sind auch die Ergebnisse der Prediction besser. 
                    \item \textbf{Anzahl zufälliger Beispiele $|I|_{rand}$:} Eine Erhöhung der Anzahl der zufälligen Beispiele hat in der durchgeführten Simulation die gleiche 
                    Auswirkung wie eine Erhöhung der Anzahl der vorangegangen Beispiele. Auch hier verbessert sich das Fehlerverhalten am Anfang sehr stark. Allgemein ist bei einer Erhöhung
                    der Anzahl der Beispiele immer zu berücksichtigen, dass damit auch der Rechenaufwand deutlich anwächst. Es erscheint mit diesen Ergebnissen sinnvoll, die 
                    Anzahl beider Beispielarten leicht zu erhöhen.
                    \item \textbf{Anzahl der Eingänge:} In dem hier betrachteten Test konnte nur eine geringe Verbesserung durch die Erhöhung der Anzahl der Eingänge festgestellt werden. 
                    Sicher ist es nicht sinnvoll nur einen Eingang zu benutzen, allerdings bestätigt sich das Ergebnis aus der Analyse in Kapitel \ref{Analyse}, 
                    dass eine Erhöhung der Anzahl der Eingänge nicht zwangsläufig
                    die Ergebnisse verbessert. Daher werden die Eingänge nur geringfügig von jeweils drei auf vier angepasst. Damit soll ein gewisser Overhead an Informationen 
                    für andere Strecken geschaffen werden.
                    \item \textbf{Anzahl der Neuronen auf einem Layer $n_{max}$:} Der Einfluss der Anzahl der Neuronen auf einem Layer ist schwierig zu bewerten, da in allen 
                    Testreihen eine Anzahl von insgesamt sechs Neuronen nicht überschritten wird. Daher wird für die passende Einstellung das Ergebnis aus der Offline-Analyse betrachtet. 
                    Dort zeigte sich, dass die Anzahl der Neuronen auf einem Layer deutlich größeren Einfluss auf die Performance hat als die Anzahl der Layer selbst. Daher wird die 
                    Anzahl der Neuronen auf einem Layer auf acht erhöht.
                    \item \textbf{Anzahl der neuen zufälligen Beispiele $\Delta|I|_{rand}$:} Eine Erhöhung der Anzahl, der in jedem Zeitschritt erneuerten zufälligen Beispiele, konnte
                    in der durchgeführten Simulation zu einer Verbesserung in Bezug auf den Fehler führen. Der minimale Fehler konnte in diesem Fall auf $10^{-6}$ reduziert werden. 
                    Ein stärkerer Wechsel der zufälligen Beispiele sorgt dafür, dass in jedem Training der Fehler stärker variiert, der durch das Netz zurück propagiert wird. Damit werden die 
                    Anforderungen an das Training erhöht. Aus diesem Grund wird die Anzahl der neuen Beispiele pro Zeitschritt auf 40 erhöht. 
                    \item \textbf{Anzahl der Predictionszeitschritte $k_{p}$:} Eine Erhöhung der Anzahl der Zeitschritte, für die die Prediction berechnet wird, führt zu einer generellen 
                    Erhöhung des Niveaus des Fehlers. Da mehr Zeitschritte ausschließlich aus dem Modell berechnet werden, erhöht sich der Fehler automatisch. Die Anforderungen an 
                    das Modell werden mit einer größeren Anzahl an Zeitschritten für die Prediction deutlich vergrößert. Zudem wächst der Rechenaufwand stark an. Daher wurde die Anzahl 
                    unverändert gelassen. 
                    \item \textbf{Reset-Interval $k_{max}$:} Die Länge des Reset-Intervalls (Anzahl der Zeitschritte vor neuer Initialisierung des letzten Layers) zeigt im Test kaum 
                    einen Einfluss bei einer Erhöhung. Um dem Training noch etwas mehr Zeit für ein gutes Ergebnis zu geben, wurde die Anzahl auf 3000 erhöht.
                    \item \textbf{Anfängliche Fehlergrenze $s_{0E}$:} Die anfängliche Fehlergrenze hat nur auf die ersten Strukturanpassungen Einfluss. Je größer sie gewählt wird, umso 
                    schneller werden die ersten Strukturanpassungen durchgeführt. Mit diesem Wert kann der Benutzer festlegen, welche Performance bereits von einfachen Strukturen zu 
                    erfüllen ist. Für die weiteren Berechnungen wurden keine Änderungen vorgenommen.
                 \end{itemize}
                 
            \subsubsection{Angepasstes System}
                Mit den Erkenntnissen aus dem vorangegangenen Abschnitt kann nun das System angepasst werden und die Simulation erneut durchgeführt werden. Die angepassten Parameter 
                sind in Tabelle \ref{Tab:ParameterAngepasst} aufgeführt. 
                \begin{table}[h]
                    \begin{tabular}{c||ccccccc}
                        %\hline
                        Parameter &Signal&Eingänge $u(t)$&Eingänge $y(t)$&$|I|_{last}$&$|I|_{rand}$&$\Delta|I|_{rand}$&$n_{max}$\\
                        %\hline
                        Wert &Step&$4$&$4$&$25$&$200$&$40$&$5$\\
                        \hline
                        \hline
                        Parameter &$k_{p}$&$s0_{E}$&$\Delta s_{E}$&$k_{max}$&&&\\
                        %\hline
                        Wert &$8$&$10^{-3}$&$0{,}8$&$3000$&&&\\
                        %\hline
                    \end{tabular}
                    \centering
                    \caption{Parameter für die angepasste Simulation (Strecke Nr.1)}
                    \label{Tab:ParameterAngepasst}
                \end{table}
                Damit konnte der Fehlerverlauf in Abbildung \ref{Abb:RealTime_Testreihe101} erzielt werden.
                \begin{figure}[h]
                    \begin{tikzpicture}
                        \begin{semilogyaxis}[xmin = 0, xmax = 40, width=16cm, height=6cm, grid=major, grid style={dashed,gray!30}, xlabel=Zeit, 
                            ylabel=Fehler, x unit= Min , y unit=] 
                            \addplot[mark=none,blue] table[x index = {0}, y index = {1}]{Daten/Auswertung_Echtzeit_Setup_101/pred_error.txt};
                        \end{semilogyaxis}
                    \end{tikzpicture}
                    \begin{tikzpicture}
                        \begin{axis}[xmin = 0, xmax = 40, width=16cm, height=4cm, grid=major, grid style={dashed,gray!30}, xlabel=Zeit, 
                            ylabel=Neuronen, x unit= Min, y unit= ,]  
                            \addplot[mark=none,red] table[x index = {0}, y index = {1}]{Daten/Auswertung_Echtzeit_Setup_101/struktur.txt};
                            %\draw (axis cs:1000,2) circle[radius=2];
                        \end{axis}
                    \end{tikzpicture}
                    \caption{Fehler und Strukturverlauf mit angepasstem System (Strecke Nr.1)}
                    \label{Abb:RealTime_Testreihe101}
                \end{figure}
                Dabei wurde eine Simulationszeit von etwa 80 Minuten betrachtet (nur die ersten 40 Minuten sind dargestellt, da sich danach keine Verbesserungen mehr ergeben). 
                Es ist zu erkennen, dass das Fehlerniveau um etwa eine Größenordnung niedriger liegt als in der 
                Referenzsimulation (siehe Abbildung \ref{Abb:RealTime_Testreihe1}). Das deutet darauf hin, dass die Anpassung der Parameter sinnvoll ist. Die Strukturanpassung an 
                sich verläuft sehr ähnlich. Auch in dieser Simulation werden insgesamt sechs Anpassungen durchgeführt, allerdings sind die Neuronen anders verteilt.
                Das Ergebnis deutet darauf hin, dass mit den angepassten Parametern in jeder Struktur das Training effizienter durchgeführt werden konnte.\\
                Um die Verbesserung des Modells zu verdeutlichen, zeigt Abbildung \ref{Abb:Modelle101} die gemessene Regelgröße und die mit drei verschiedenen Netzen berechneten Verläufe.
                \begin{figure}[h]
                    \begin{tikzpicture}
                        \begin{axis}[xmin = 0, xmax = 60, width=16cm, height=6cm, grid=major, grid style={dashed,gray!30}, xlabel=Zeit, 
                            ylabel=Regelgröße $y(t)$, x unit= s , y unit=, legend entries={Messung,nach 1\,Min,nach 13\,Min,nach 30\,Min}] 
                            \addplot[mark=none,black] table[x index = {0}, y index = {1}]{Daten/Auswertung_Echtzeit_Setup_101/Modellberechnung.txt};
                            \addplot[mark=none,red] table[x index = {0}, y index = {3}]{Daten/Auswertung_Echtzeit_Setup_101/Modellberechnung.txt};
                            \addplot[mark=none,blue] table[x index = {0}, y index = {5}]{Daten/Auswertung_Echtzeit_Setup_101/Modellberechnung.txt};
                            \addplot[mark=none,green] table[x index = {0}, y index = {7}]{Daten/Auswertung_Echtzeit_Setup_101/Modellberechnung.txt};
                        \end{axis}
                    \end{tikzpicture}
                    \caption{Verlauf der Regelgröße aus Messung und Modellberechnungen mit Modellen zu unterschiedlichen Zeitpunkten (Strecke Nr.1)}
                    \label{Abb:Modelle101}
                \end{figure}
                Dabei zeigt sich deutlich, dass die Modelle mit der Zeit immer besser werden. Das Netz nach etwa einer Minute (rot) zeigt eine sehr deutliche Abweichung zur Messung (schwarz).
                Nach etwa 13 Minuten (blau) ist das Ergebnis bereits verbessert. Auch nach etwa 30 Minuten (grün), bei der letzten Strukturanpassung, liegt die Modellberechnung
                nochmal näher an der Messung. Diese Ergebnisse zeigen eindeutig, dass das System in der Lage ist, ein Modell über die Zeit sinnvoll 
                anzupassen und die Modellqualität zunehmend zu verbessern.
                
            \subsubsection{Alterung der Strecke}
                Reale Übertragungsglieder können sich im Laufe der Zeit z.B. durch Alterung verändern. Daher sollte sich das Modell auch mit der Zeit an die Alterung anpassen.
                Nur so kann das Modell dauerhaft gleich gute Ergebnisse liefern. Um diese Eigenschaft zu überprüfen, wird der Strecke Nr.1 eine künstliche Alterung hinzugefügt. 
                Der Parameter $c_2$ verändert sich linear mit der Zeit.
                \begin{figure}[h]
                    \begin{tikzpicture}
                        \begin{axis}[xmin = 0, xmax = 60, width=16cm, height=6cm, grid=major, grid style={dashed,gray!30}, xlabel=Zeit, 
                            ylabel=$u(t)$/$y(t)$, x unit= s , y unit=, legend entries={$u(t)$,$y(t)$}] 
                            \addplot[mark=none,blue] table[x index = {0}, y index = {1}]{Daten/Testsignale/Skript_Alterung.txt};
                            \addplot[mark=none,red] table[x index = {0}, y index = {2}]{Daten/Testsignale/Skript_Alterung.txt};
                        \end{axis}
                    \end{tikzpicture}
                    \caption{Verlauf von Stellgröße und Regelgröße der Strecke Nr.1 mit linear alterndem Parameter $c_2$}
                    \label{Abb:Alterung}
                \end{figure}
                Der Verlauf der Stellgröße und der Regelgröße für die Strecke mit veränderlichem Parameter ist in Abbildung 
                \ref{Abb:Alterung} dargestellt.\\
                Auch mit den erstellten Messdaten aus der gealterten Strecke wurde eine Simulation von etwa 30 Minuten durchgeführt. Die Verläufe des Fehlers und der Struktur sind
                in Abbildung \ref{Abb:RealTime_Alterung} dargestellt.
                \begin{figure}[h]
                    \begin{tikzpicture}
                        \begin{semilogyaxis}[xmin = 0, xmax = 30, width=16cm, height=6cm, grid=major, grid style={dashed,gray!30}, xlabel=Zeit, 
                            ylabel=Fehler, x unit= Min , y unit=] 
                            \addplot[mark=none,blue] table[x index = {0}, y index = {1}]{Daten/Auswertung_Echtzeit_Setup_102/pred_error.txt};
                        \end{semilogyaxis}
                    \end{tikzpicture}
                    \begin{tikzpicture}
                        \begin{axis}[xmin = 0, xmax = 30, width=16cm, height=4cm, grid=major, grid style={dashed,gray!30}, xlabel=Zeit, 
                            ylabel=Neuronen, x unit= Min, y unit= ,]  
                            \addplot[mark=none,red] table[x index = {0}, y index = {1}]{Daten/Auswertung_Echtzeit_Setup_102/struktur.txt};
                            %\draw (axis cs:1000,2) circle[radius=2];
                        \end{axis}
                    \end{tikzpicture}
                    \caption{Fehler und Strukturverlauf für die Identifikation von Strecke Nr.1 mit künstlicher Alterung}
                    \label{Abb:RealTime_Alterung}
                \end{figure}
                Für diese Simulation zeigt sich ebefalls eine deutliche Verbesserung des Modells mit der Zeit. Daraus lässt sich schließen, dass das System auch 
                in der Lage ist, den Alterungsprozess mit zu erlernen. Die Genauigkeit des Modells wird durch den veränderlichen Parameter in der Strecke nicht 
                verändert. Das System ist in der Lage das Modell an den Alterungsprozess anzupassen und so zuverlässige Ergebnisse zu liefern.\\
                Die Anpassung an die Alterung lässt sich besonders gut in der Betrachtung verschiedener Modelle über die Zeit feststellen. Abbildung \ref{Abb:Modelle102} zeigt den 
                Verlauf der Messung und die mit den Modellen berechneten Verläufe der Regelgröße. 
                \begin{figure}[h]
                    \begin{tikzpicture}
                        \begin{axis}[xmin = 0, xmax = 60, width=16cm, height=6cm, grid=major, grid style={dashed,gray!30}, xlabel=Zeit, 
                            ylabel=Regelgröße, x unit= s , y unit=, legend entries={Messung,nach 1\,Min,nach 2\,Min,nach 7\,Min}] 
                            \addplot[mark=none,black] table[x index = {0}, y index = {1}]{Daten/Auswertung_Echtzeit_Setup_102/Modellberechnung.txt};
                            \addplot[mark=none,red] table[x index = {0}, y index = {3}]{Daten/Auswertung_Echtzeit_Setup_102/Modellberechnung.txt};
                            \addplot[mark=none,blue] table[x index = {0}, y index = {4}]{Daten/Auswertung_Echtzeit_Setup_102/Modellberechnung.txt};
                            \addplot[mark=none,green] table[x index = {0}, y index = {5}]{Daten/Auswertung_Echtzeit_Setup_102/Modellberechnung.txt};
                        \end{axis}
                    \end{tikzpicture}
                    \caption{Verlauf der Regelgröße aus Messung und Modellberechnungen mit Modellen zu unterschiedlichen Zeitpunkten (Strecke Nr.1 mit künstlicher Alterung)}
                    \label{Abb:Modelle102}
                \end{figure}
                Dabei wurden die Modelle nach einer Minute, nach zwei Minuten und nach etwa sieben Minuten ausgewählt. Danach ergeben sich nur noch geringfügige Verbesserungen, 
                die vor allem bei sehr langen Berechnungen mit dem Modell positive Auswirkungen haben. Bei den drei aufgeführten Modellen zeigt sich, wie das System den 
                Alterungsvorgang zunehmend gut abbilden kann. Die Ergebnisse weisen darauf hin, dass das System auch zur Identifikation von Strecken verwendet werden kann, 
                die über die Zeit veränderlich sind.          
                                        
        \subsection{Lorenz-Attraktor}
            Das im vorangegangenen Abschnitt eingestellte System soll nun an einer weiteren Strecke getestet werden. Dafür wird der sogenannte Lorenz-Attraktor betrachtet.
            \subsubsection{Streckenbeschreibung}
                Der Lorenz-Attraktor wird durch ein System aus drei gekoppelten Differentialgleichungen beschrieben:
                \begin{align}
                    \begin{split}
                    &\dot{x}_1 = a\,(x_2 - x_1)\\
                    &\dot{x}_2 = b\,x_1 −x_2 −x_1\,x_3\\
                    &\dot{x}_3 = − b\,x_3 + x_1\,x_2
                    \end{split}
                \end{align}
                \begin{flushright}
                    (Formel aus \cite{prill})
                \end{flushright}
                Eine sehr ausführliche Beschreibung des Systems und dessen Eigenschaften ist in \cite{prill} zu finden. Auf die genaue Herleitung wird an dieser Stelle verzichtet. 
                Das System wird zur Beschreibung von Zuständen in der Erdatmosphäre verwendet, ebenso wie für die Beschreibung von bestimmten Phänomenen der Konvektion. Der Begriff 
                Attraktor wird verwendet, da Lorenz bei der Arbeit an dem System feststellte, dass sich für bestimmte Parameter eine unregelmäßige Oszillation einstellt (vgl. \cite{prill}).
                Da dieses chaotische Verhalten höchste Anforderungen an das Identifikationssystem stellt, wird hier explizit der chaotische Fall betrachtet. 
                Dafür werden die Parameter $a = 10$, $b = 28$ und $c = 8/3$ gewählt (übernommen von \cite{wiki}). Um zusätzlich einen Kontrolleingriff zu ermöglichen und einen skalaren 
                Ausgangswert zu erhalten, wird das System leicht modifiziert:
                \begin{align}
                    \begin{split}
                    &\dot{x}_1 = 10\,(x_2 - x_1)+u\\
                    &\dot{x}_2 = 28\,x_1 −x_2 −x_1\,x_3\\
                    &\dot{x}_3 = − \frac{8}{3}\,x_3 + x_1\,x_2
                    \end{split}\\
                    &y = \frac{x_1+x_2+x_3}{100}
                \end{align}
                Diese Veränderung erlaubt die einfache Betrachtung als SISO-System wie bereits für Strecke Nr.1. Die Ergebnisse sind somit leichter darstellbar, als wenn 
                alle Variablen des Systems betrachtet werden müssen.
                
            \subsubsection{Unangepasstes System}
                Das Ziel ist es eine Parameterkombination zu finden, die für möglichst viele Strecken zuverlässig gute Ergebnisse ermöglicht. Daher wird nun das für die 
                Strecke Nr.1 eingestellte System unverändert auf das Lorenz-System angewendet. Die Parameter sind in Tabelle \ref{Tab:ParameterAngepasst} aufgelistet. 
                Abbildung \ref{Abb:RealTime_Testreihe201} zeigt den Fehler- und Strukturverlauf für die durchgeführte Simulation.
                \begin{figure}[h]
                    \begin{tikzpicture}
                        \begin{semilogyaxis}[xmin = 0, xmax = 40, width=16cm, height=6cm, grid=major, grid style={dashed,gray!30}, xlabel=Zeit, 
                            ylabel=Fehler, x unit= Min , y unit=] 
                            \addplot[mark=none,blue] table[x index = {0}, y index = {1}]{Daten/Auswertung_Echtzeit_Setup_201/pred_error.txt};
                        \end{semilogyaxis}
                    \end{tikzpicture}
                    \begin{tikzpicture}
                        \begin{axis}[xmin = 0, xmax = 40, width=16cm, height=4cm, grid=major, grid style={dashed,gray!30}, xlabel=Zeit, 
                            ylabel=Neuronen, x unit= Min, y unit= ,]  
                            \addplot[mark=none,red] table[x index = {0}, y index = {1}]{Daten/Auswertung_Echtzeit_Setup_201/struktur.txt};
                            %\draw (axis cs:1000,2) circle[radius=2];
                        \end{axis}
                    \end{tikzpicture}
                    \caption{Fehler und Strukturverlauf für Lorenz-System mit vorhandenen Parametern}
                    \label{Abb:RealTime_Testreihe201}
                \end{figure} 
                Es sind wieder nur die ersten 40 von 80 Minuten der Simulation dargestellt, da danach keine deutliche Veränderung mehr auftritt. 
                Es fällt auf, dass insgesamt das Fehlerniveau deutlich höher ist als für Strecke Nr.1. Dies ist auf das schwierige und chaotische Verhalten des Lorenz-Attraktors 
                zurückzuführen. Eine Verbesserung über die Zeit ist jedoch weiterhin festzustellen. Durch das höhere Fehlerniveau treten zudem weniger Strukturanpassungen auf. 
                Es scheint mit diesen Einstelllungen schwierig zu sein, das Modell stetig zu verbessern. Eine Verbesserung kann möglicherweise durch eine Anpassung der anfänglichen 
                Fehlergrenze $s_{0E}$ erreicht werden.\\
                Abbildung \ref{Abb:Modelle201} zeigt zwei unterschiedliche Modelle jeweils kurz vor der Strukturanpassung. 
                Hier ist zwischen den beiden Modellen zwar eine leichte Verbesserung über die Zeit zu erkennen, allerdings bei weitem nicht so deutlich wie für Strecke Nr.1.
                Dies deutet ebenfalls darauf hin, dass die gewählten Parameter für das System noch nicht passend sind.  
                \begin{figure}[H]
                    \begin{tikzpicture}
                        \begin{axis}[xmin = 0, xmax = 60, width=16cm, height=6cm, grid=major, grid style={dashed,gray!30}, xlabel=Zeit, 
                            ylabel=Regelgröße, x unit= s , y unit=, legend entries={Messung,nach 2\,Min,nach 20\,Min}] 
                            \addplot[mark=none,black] table[x index = {0}, y index = {1}]{Daten/Auswertung_Echtzeit_Setup_201/Modellberechnung.txt};
                            \addplot[mark=none,red] table[x index = {0}, y index = {2}]{Daten/Auswertung_Echtzeit_Setup_201/Modellberechnung.txt};
                            \addplot[mark=none,blue] table[x index = {0}, y index = {3}]{Daten/Auswertung_Echtzeit_Setup_201/Modellberechnung.txt};
                        \end{axis}
                    \end{tikzpicture}
                    \caption{Verlauf der Regelgröße aus Messung und Modellberechnungen mit Modellen zu unterschiedlichen Zeitpunkten (Lorenz-Attraktor)}
                    \label{Abb:Modelle201}
                \end{figure}

            \subsubsection{Angepasstes System}
                Um zu überprüfen, ob die Ergebnisse für das Lorenz-System noch zu verbessern sind, wurden diverse Parameterkombinationen ausprobiert. Dabei ergaben sich die besten Ergebnisse
                für die Referenzparameter aus Tabelle \ref{Tab:ParameterRef}.   
                \begin{figure}[h]
                    \begin{tikzpicture}
                        \begin{axis}[xmin = 0, xmax = 60, width=16cm, height=6cm, grid=major, grid style={dashed,gray!30}, xlabel=Zeit, 
                            ylabel=Regelgröße, x unit= s , y unit=, legend entries={Messung,nach 12\,Min,nach 50\,Min}] 
                            \addplot[mark=none,black] table[x index = {0}, y index = {1}]{Daten/Auswertung_Echtzeit_Setup_11/Modellberechnung.txt};
                            \addplot[mark=none,red] table[x index = {0}, y index = {3}]{Daten/Auswertung_Echtzeit_Setup_11/Modellberechnung.txt};
                            \addplot[mark=none,blue] table[x index = {0}, y index = {5}]{Daten/Auswertung_Echtzeit_Setup_11/Modellberechnung.txt};
                        \end{axis}
                    \end{tikzpicture}
                    \caption{Verlauf der Regelgröße aus Messung und Modellberechnungen mit Modellen zu unterschiedlichen Zeitpunkten (Lorenz-Attraktor, angepasste Parameter)}
                    \label{Abb:Modelle11}
                \end{figure}
                Abbildung \ref{Abb:Modelle11} zeigt den gemessenen Verlauf der Stellgröße zusammen mit zwei Verläufen, die mit Modellen berechnet wurden. Dabei zeigt sich, dass in 
                dieser Konfiguration die Ergebnisse über die Zeit deutlich besser werden. Der Verlauf mit dem Modell nach etwa 50 Minuten liegt deutlich näher an der Messung als 
                der Verlauf mit dem Modell nach etwa zwölf Minuten.\\
                Das Ergebnis zeigt, dass die Parameter einen starken Einfluss auf die Funktionsweise des Systems haben. Mit den hier betrachteten Versuchen konnte 
                noch keine Parametereinstellung gefunden werden, die durchweg positive Ergebnisse liefert. Es ist auch möglich, dass eine für alle Strecken passende 
                Parametereinstellung nicht existiert. Im Rahmen dieser Arbeit war der Zeitaufwand einer ausführlichen Parameteranalyse jedoch zu groß.   
            