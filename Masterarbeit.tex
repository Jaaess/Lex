\documentclass[12pt,a4paper,listof=totocnumbered, bibliography=totocnumbered,oneside]{scrbook}

\usepackage[ngerman]{babel} 						%Deutsche Trennung
\usepackage[utf8]{inputenc}						    %Zeichensatz
%\usepackage[numbers]{natbib}					    %Zusätzliche Simbole
\usepackage[backend=biber]{biblatex} 
\bibliography{Literatur}
									
\usepackage{amsmath}        					    %MathePacket
\usepackage{amsfonts}						        %Zusätliche Mathezeichen
\usepackage{amssymb}							    %Zusätzliche Mathezeichen wie Pfeile 
\usepackage[]{graphicx}					            %Packet für Bilder

\usepackage[headsepline,footsepline,plainfootsepline,markcase=upper]{scrlayer-scrpage}
\usepackage{tabularx}							    %Tabellen
\usepackage{tabu}
\usepackage{geometry}							    %Seitengeometrie
\usepackage[onehalfspacing]{setspace}			    %Zeilenabstand
\usepackage[printonlyused]{acronym}			    %Abkürzungen
\usepackage{subfig}								    %mehrere Bilde
%\usepackage{floatflt}							    %Objekte von Text umfließen lassen
\usepackage{float}								    %Positionierung von Objeten
\usepackage{wrapfig}							    %Objekte von Text umfließen lassen
\usepackage[usenames,dvipsnames]{color}		    %Farben für Latex
\usepackage{colortbl}							    %Farben für Tabellen
\usepackage{paralist}							    %Auflistungen
\usepackage{array}							        %Tabllen und mAtrizen in Matheumgebung
\usepackage{titlesec}							    %Überschriften verändern
\usepackage{parskip}							    %Absätze einstellen
%\usepackage{picins}							    %Umfließen vom Text
\usepackage[subfigure,titles]{tocloft}			    %Inhaltsverzeichnis formatieren
\usepackage[breaklinks=true,pdfpagelabels=true]{hyperref}%Verlinkungen erstellen
\usepackage{listings}							    %Quellcode einfügen
\usepackage{siunitx}							    %SI Schreibweise
\usepackage{chngcntr}							    %figurennumerierung
\usepackage[hyphenbreaks]{breakurl}			    %Zeilenumbruch links
%\usepackage{multibib}							    %mehrere literaturverzeichnisse
%\usepackage{times}								    %Times new roman
\usepackage{longtable}							    %Tabellen über mehrere Seiten
\usepackage{framed}								    %Rahmen
\usepackage{diagbox}							    %Diagonalboxen für Tabellen
\usepackage{xcolor}								    %noch mehr Farben						
\usepackage{nicefrac}							    %schöne Brüche
\usepackage{textpos}							    %Positionierung von Text
\usepackage{multicol}

\usepackage{booktabs}
\usepackage{pgfplots}
\usepackage{pgfplotstable}
\usepackage{tikz}
\usepgfplotslibrary{units}
%\pgfplotsset{compat=1.10}
\usepackage{caption}

%Eigene Farben
\definecolor{lightgray}{gray}{0.90}
\definecolor{tableheadergray}{gray}{0.95}
\definecolor{limegreen}{HTML}{E1F5A9}
\definecolor{tured}{HTML}{C50E1F}
\definecolor{tugrey}{HTML}{717171}

\newenvironment{cmssb}{\fontfamily{cmss}\fontseries{bx}\fontshape{n}\selectfont}{\par}
\newenvironment{cmss}{\fontfamily{cmss}\fontseries{m}\fontshape{n}\selectfont}{\par} 


%Geometrie der Seite Einrichten
\geometry{a4paper, top=20mm, left=25mm, right=15mm, bottom=20mm, headsep=5mm, footskip=10mm}
%\areaset{15cm}{20cm}

\lohead[]{}
\cohead[]{}
\rohead[]{}
\lehead[]{}
\cehead[]{}
\rehead[]{}
\lofoot[]{}
\cofoot[Seite \pagemark]{Seite \pagemark}
\rofoot[]{}
\lefoot[]{}
\cefoot[]{Seite \pagemark}
\refoot[]{}

%\ModifyLayer[addvoffset=-2pt]{scrheadings.foot.above.line}
\setkomafont{pagehead}{\normalfont\color{tugrey}}
\setkomafont{pagefoot}{\small\normalfont\color{tugrey}}
\setkomafont{headsepline}{\color{tugrey}}
\setkomafont{footsepline}{\color{tugrey}}

\makeatletter 
\renewcommand{\l@figure}{\@dottedtocline{1}{1.5em}{3em}} 
\renewcommand{\l@table}{\@dottedtocline{1}{1.5em}{3em}} 

%Nummerierungen
\numberwithin{figure}{chapter}			    %Nummerierung der figuren
\numberwithin{equation}{chapter}		        %Nummerierung der formeln

% Abstände Überschrift
\titlespacing{\section}{0pt}{12pt plus 4pt minus 2pt}{-6pt plus 2pt minus 2pt}
\titlespacing{\subsection}{0pt}{12pt plus 4pt minus 2pt}{-6pt plus 2pt minus 2pt}
\titlespacing{\subsubsection}{0pt}{12pt plus 4pt minus 2pt}{-6pt plus 2pt minus 2pt}

\pagenumbering{Roman}
\renewcommand\thechapter{\Roman{chapter}}
\renewcommand\thesection{\roman{section}}
%-------------------
\begin{document}
    \begin{titlepage}
    \newgeometry{left=24mm, right=8mm, top=10mm, bottom=13mm}
	\begin{textblock*}{0mm}(118mm,0mm)
			\includegraphics[width=46mm]{Abbildungen/Deckblatt/TU_Logo_lang_4c_rot.pdf}  
	\end{textblock*} 

\vspace*{4cm}
    
    \begin{cmssb}
    	\begin{flushleft}
    		%\textcolor{tured}{\rule{13cm}{.4pt}}\\
    		\vspace*{10pt} 
    		\textsf{\fontsize{14}{0}\selectfont TU Berlin Fachgebiet Mess- und Regelungstechnik}\\
			\vspace*{10pt} 
			\textsf{\fontsize{30}{0}\selectfont Masterarbeit}\\
			\vspace*{10pt}
			\textcolor{tured}{\textsf{\fontsize{14}{0}\selectfont Titel der Arbeit}}
			%\textcolor{tured}{\rule{13cm}{.4pt}}\\
		\end{flushleft}
	\end{cmssb}
 
\begin{textblock*}{60mm}(120mm,10mm)
    \begin{flushleft}
    	\begin{cmss}
    		\begin{normalsize}		
    			\textcolor{tugrey}{Vorgelegt von:}\\
				Jawhar Ben Hadj M'Barek\\
				Franz Mehring Platz. 3\\
				10243 Berlin \linebreak\linebreak
				Mat.-Nr. 738082\\
				s63338@beuth-hochschule.de
				\linebreak\linebreak\linebreak
				\textcolor{tugrey}{Erstprüfer:}\\
				Prof. Rudibert King\\
				\textcolor{tugrey}{Zweitprüfer:}\\
				M. Sc. Florian Arnold\\
				\textcolor{tugrey}{Abgabe:}\\
				26. November 2018\linebreak\linebreak
			\end{normalsize}	
		\end{cmss}
    \end{flushleft}    
\end{textblock*} 	

\end{titlepage}

\newpage

\thispagestyle{empty}

    %\pagebreak
%leer
%\thispagestyle{empty}
%\pagebreak
\section*{Selbstständigkeitserklärung}
\vspace{5cm}
\begin{framed}
Hiermit erkläre ich, dass ich die vorliegende Arbeit selbstständig und eigenhändig sowie ohne unerlaubte
fremde Hilfe und ausschließlich unter Verwendung der aufgeführten Quellen und Hilfsmittel angefertigt habe.\\

Berlin, den\\

............................................................................................................\\
Unterschrift
\end{framed}
\newpage


    \setcounter{page}{2}
    \chapter{Zusammenfassung/Abstract}
    \section{Zusammenfassung}
       \hspace{1cm} Der PKW ist das am häufigsten benutzte Fahrzeug in Unternehmen wie beispielsweise Kurier-, Taxi- oder Pflegedienste. PKW Diebstähle und die Kosten für Benzin und Instandhaltung von PKW's sind so immens gestiegen, dass viele Unternehmer genau dadurch der PKW Überwachung wesentlich mehr Aufmerksamkeit und Interesse schenken als ohne. \newline

So viele PKW's überwachen wie man möchte, ohne dass man dabei ist, ganz bequem vom Büro aus. Nicht nur gegen Diebstahl schützen, sondern ebenso Alarm auslösen bei Beschädigungen oder offenen Türen. Außerdem ist auch interessant eine Kostenkontrolle und -analyse über die hohen Benzinkosten zu erhalten. Weiterhin ist es wichtig, dass noch eine genaue Aufstellung der einzelnen Fahrer erfolgt. Welche Route der Fahrer genommen hat sowie die Start- und Stoppzeiten werden detailliert aufgelistet. 

    \section{Abstract}

    Automotive knowledge combined with it Industry.
     the automotive market today is more driven by the IT industry by new use cases that are coming out from the IT world.
     People want to do shopping from inside their cars. they want to use services that are connected to the IT word. The challenge for the automotive manufacturers is to compete with a market that's driven by smart devices, which develop in very fast way.
     
    
Monitoring the water temperature, the motor rotational speed, the oil level and its temprature and other important information to make sure you can identify any errors or mistakes that may happen on the way, and resolve them on time to save your company from potential business loss.


    
Act quick and be in control

Once in a while unexpected situations occur, in which case it's important to respond quickly and adjust the plan. Live map and GPS tracking will be extremely useful when finding the nearest vehicle to help other driver nearby and assisting with directions if any driver is off the road.

GPS tracking and live data receiving wouldn't be possible without accessing to the responsable telematics devices or sensors, which deliver informations about the different equipements's status in the vehicle, therefore we have to find a way how to extract the needed informations, than how to transfer them through a cloud solution and visualize them in real time.
Automotive industry have to speed up and comes with new technical solutions to integrate the consumer eloctronics inside the vehicle ones.

IOT is one of the promising thechnologies that the automotive industry should integrate in the vehicle, until now it is limited for tradional use cases like preparing car that could be controlled out of the cloud, out of the internet so that technical maintenance and intervention can be scheduled. Another use cases are related to autonomous driving of cars under safe operative systems.

 

        Englischer Text Englischer Text Englischer Text Englischer Text Englischer Text Englischer Text Englischer Text Englischer Text             
        Englischer Text Englischer Text Englischer Text Englischer Text Englischer Text Englischer Text Englischer Text Englischer Text 
        Englischer Text Englischer Text Englischer Text Englischer Text Englischer Text Englischer Text Englischer Text Englischer Text 
        Englischer Text Englischer Text Englischer Text Englischer Text Englischer Text Englischer Text Englischer Text Englischer Text 
        Englischer Text Englischer Text Englischer Text Englischer Text Englischer Text Englischer Text Englischer Text Englischer Text
        
    \tableofcontents
    %\lohead[]{llll}
\chapter{Nomenklatur}
    \section{Formelzeichen}
    \begin{longtable}[l]{p{3cm}|p{12cm}}
        Formelzeichen & Beschreibung\\
        \hline
        $A$ &Ausgabeschicht/Ruhelage\\
        $c$ & Koeffizienten im Polynom/Koeffizienten in Teststrecke \\
        $\vec{c}$ &Center-Vektor\\
        $\vec{d}$ & Abstand zwischen zwei Vektoren\\
        $E$ & Eingabeschicht/Fehler\\
        $e$ & Quadratischer Fehler\\
        $f$ & Funktion/Frequenz\\ 
        $flops$ & Fließkommaoperationen \\
        $g$ & Funktion/Parameter\\
        $H$ & Verborgene Schicht bzw. Layer\\
        $h$ & Funktion\\
        $I$ & Netzeingabe\\
        $|I|$ & Anzahl Beispiele\\
        $\vec{i}$ & Input eines einzelnen Neurons\\
        $J$ & Jacobimatrix\\
        $k$ & Anzahl \\
        $m$ & Anzahl der Ausgänge/ganzzahliges Vielfaches\\
        $N$ & Menge aller Neuronen\\
        $n$ & Anzahl Neuronen/Anzahl Trainingsbeispiele\\
        $O$ & Netzausgabe\\
        $o$ & Output eines einzelnen Neurons\\
        $P$ & Performance\\
        $p$ & Anzahl der Beispiele die erneuert werden\\
        $q$ & Grad des Polynoms/Anzahl der Layer\\
        $r$ & Zeitauflösung\\
        $s$ & Grenze bzw. Schranke\\
        $T$ & Ausgabemuster/Teachinginput\\
        $t$ & Zeit bzw. Zeitpunkt\\
        $u$ & Stellgröße\\
        $V$ & Menge aller Verbindungen\\
        $v$ & Gewicht/Polynom\\
        $W$ & Matrix aller Gewichte\\
        $\vec{w}$ & Eingangsgewichte eines einzelnen Neurons\\
        $w$ & Gewicht\\
        $Y$ & Messwert\\
        $y$ & Regelgröße\\
        $\vec{Z}$ & Innere Zustände eines Neurons\\
        &\\
        $\alpha$ & Parameter \\
        $\beta$ & Parameter \\
        $\Delta$ & Änderung \\
        $\delta$ & Gewichtsänderung\\
        $\eta$ & Lernrate\\
        $\Theta$ & Parameter\\
        $\mu$ & Regularisierungsfaktor\\
        $\rho$ & Korrelationswert\\ 
        $\tau$ & Integrationsvariable\\
        $\nabla$ & Nablaoperator \\
       
    \end{longtable}

    \section{Indizes}
    
    \begin{longtable}[l]{p{3cm}|p{12cm}}
        Index & Beschreibung\\
        \hline
        $A$ & Aktivierung\\
        $alt$ & alt/aus vergangenem Zeitschritt\\
        $i$ & Zählindex\\
        $j$ & Zählindex\\
        $k$ & Zählindex\\
        $last$ & Vergangene Werte\\
        $Modell$/$M$ & Größen die aus dem Modell berechnet wurden\\
        $max$ & Maximal\\
        $o$ & Ausgabe\\
        $P$ & Propagierung\\
        $p$ & Prediction\\
        $^p$ & Ein bestimmtes Trainingsmuster\\
        $q$ & Zählindex\\
        $rand$ & Zufällig\\  
        $s$ & Statisch\\
        $T$ & Teachinginput \\
        $Train$ & Training\\
        $Val$ & Validation\\            
        
        
    \end{longtable}

    \newpage
    %--------------Einstellungen Hauptteil--------------------
    \lohead[]{}
    \cohead[]{}
    \rohead[]{\headmark}
    \lehead[]{\headmark}
    \cehead[]{}
    \rehead[]{}
    \lofoot[]{}
    \cofoot[Seite \pagemark]{Seite \pagemark}
    \rofoot[]{}
    \lefoot[]{}
    \cefoot[]{Seite \pagemark}
    \refoot[]{}
    \pagenumbering{arabic}
    \setcounter{page}{1}
    \setcounter{chapter}{0}
    \renewcommand\thechapter{\arabic{chapter}}
    \renewcommand\thesection{\thechapter.\arabic{section}}
    %-------------Hauptteil------------------------------------
    \chapter{Einleitung}
    Einleitung Einleitung Einleitung Einleitung Einleitung Einleitung Einleitung Einleitung
    \section{Zielsetzung}
    Einleitung Einleitung Einleitung Einleitung Einleitung Einleitung Einleitung Einleitung

    
    
    \chapter{Ein Kapitel}
    Text Text Text Text Text Text Text Text Text Text Text Text Text Text Text Text Text Text Text Text Text Text Text Text 
    Text Text Text Text Text Text Text Text Text Text Text Text Text Text Text Text Text Text Text Text Text Text Text Text
    Text Text Text Text Text Text Text Text Text Text Text Text Text Text Text Text Text Text Text Text Text Text Text Text
    Text Text Text Text Text Text Text Text Text Text Text Text Text Text Text Text Text Text Text Text Text Text Text Text
    \section{Section}
        Eine Section
        \subsection{Subsection}
        
    \section{Section2}
        Eine Subsection
        \newpage
        Eine neue Seite
        \newpage 
        Noch eine neue Seite
        \newpage    
        und noch eine neue Seite
    \chapter{Ein anderes Kapitel}
    Hier ein anderes Kapitel
    Viele Zitate: \cite{patterson} \cite{krizhevsky} \cite{matlab} \cite{pitts} \cite{lawrence} \cite{miesbach}
    \section{Section}
        Eine Section
        \subsection{Subsection}
            \begin{figure}[h]
                \includegraphics[scale=0.2]{Abbildungen/Kapitel2/Kangoroo.png}
                \centering
                \caption{Ein Kangoroo}
                \label{Abb:Kangoroo}   
            \end{figure}  
             \begin{table}[h]
                \begin{tabular}{ccc}
                      \hline
                      Spalte1 & Spalte2 & Spalte3\\                      
                      \hline
                      1 & 2 & 3\\
                      \hline
                \end{tabular}
                \centering
                \caption{Eine Tabelle}
                \label{Tab:Tabelle1}
            \end{table}
 
        
        
    \section{Section2}
        Eine Subsection
        \newpage
        Eine neue Seite
        \newpage 
        Noch eine neue Seite
        \newpage    
        und noch eine neue Seite
    \chapter{Ein neues Kapitel}
    Hier ein neues Kapitel
    Viele Zitate: \cite{patterson} \cite{krizhevsky} \cite{matlab} \cite{pitts} \cite{lawrence} \cite{miesbach}
    \section{Section}
        Eine Section
        \subsection{Subsection}
            \begin{figure}[h]
                \includegraphics[scale=0.2]{Abbildungen/Kapitel3/Big-architecture.png}
                \centering
                \caption{das Architekturbild}
                \label{Abb:Architekturbild}   
            \end{figure}  
             \begin{table}[h]
                \begin{tabular}{ccc}
                      \hline
                      Spalte1 & Spalte2 & Spalte3\\                      
                      \hline
                      1 & 2 & 3\\
                      \hline
                \end{tabular}
                \centering
                \caption{Variation über Zeit}
                \label{Tab:Tabelle3}
            \end{table}
            
Hier eine neue Tabelle   
            
            \begin{table}[h]
                \begin{tabular}{ccc}
                      \hline
                      Spalte1 & Spalte2 & Spalte3\\                      
                      \hline
                      1 & 2 & 3\\
                      \hline
                \end{tabular}
                \centering
                \caption{new added Tabelle}
                \label{Tab:New added}
            \end{table}
 
        
        
    \section{Section2}
        Eine Subsection
        \newpage
        Eine neue Seite
        \newpage 
        Noch eine neue Seite
        \newpage    
        und noch eine neue Seite
    \chapter{Ein neues Kapitel}
    Hier ein neues Kapitel
    Viele Zitate: \cite{patterson} \cite{krizhevsky} \cite{matlab} \cite{pitts} \cite{lawrence} \cite{miesbach}
    \section{Section}
        Eine Section
        \subsection{Subsection}
        
        
Hier ist ein Bild:        
            \begin{figure}[h]
                \includegraphics[scale=0.2]{Abbildungen/Kapitel4/Big-architecture.png}
                \centering
                \caption{Affe Bild}
                \label{Abb:Affe}   
            \end{figure}  
            
Hier ist eine Tabelle:            
            
             \begin{table}[h]
                \begin{tabular}{ccc}
                      \hline
                      Spalte1 & Spalte2 & Spalte3\\                      
                      \hline
                      1 & 2 & 3\\
                      \hline
                \end{tabular}
                \centering
                \caption{Quadratewurzel Skalierung}
                \label{Tab:Quadratewurzel Skalierung}
            \end{table}
            
Hier eine neue Tabelle:   
            
            \begin{table}[h]
                \begin{tabular}{cccc}
                      \hline
                      Spalte1 & Spalte2 & Spalte3 & jbjvh\\                      
                      \hline
                      1 & 2 & 3 & 4\\
                      \hline
                \end{tabular}
                \centering
                \caption{Logaritmische Skalierung}
                \label{Tab:Logaritmische Skalierung}
            \end{table}
 
        
        
    \section{Section2}
        Eine Subsection
        \newpage
        Eine neue Seite
        \newpage 
        Noch eine neue Seite
        \newpage    
        und noch eine neue Seite
    \chapter{Fazit}
    Hier schrieben wie gut alles war.

    %-------------Einstellungen Ende---------------------------
    %\pagenumbering{Roman}
    %\setcounter{page}{10}
    \setcounter{chapter}{2}
    \renewcommand\thechapter{\Roman{chapter}}
    \renewcommand\thesection{\roman{section}}
    %-------------Ende-----------------------------------------
    \input{Literaturverzeichnis}
%    \lohead[]{a}
%    \cohead[]{b}
%    \rohead[]{c}
%    \lehead[]{a2}
%    \cehead[]{b2}
%    \rehead[]{c2}
%    \lofoot[]{aa}
%    \cofoot[Seite \pagemark]{Seite \pagemark}
%    \rofoot[]{cc}
%    \lefoot[]{aa2}
%    \cefoot[]{bb2}
%    \refoot[]{cc2}
    \newpage
    \listoffigures
    \newpage
    \listoftables
    \chapter{Anhang}
    Das hier ist der Anhang
    \section{Section}
    \subsection{Subsection}

\end{document}
